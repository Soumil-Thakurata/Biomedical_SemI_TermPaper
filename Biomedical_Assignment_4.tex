\documentclass[11pt]{article}

\begin{document}

NAME: Soumil Thakurata

Roll No: 21111064

Stream: BIOMEDICAL ENGINEERING

Subject: Basic Biomedical Engineering

Topic: Assignment 4

\pagebreak


\section*{\centering Disruptive Innovations in Healthcare}

A disruptive innovation is defined as a new product, service or business model when it helps create a new market, eventually disrupting existing markets and displacing previous technologies. This term was first coined by Clayton Christensen in 1995. Disruptive technologies have been predominantly noticed in the computing, photography, telecommunications, and retail sectors. The following two features have been consistently noticed regarding disruptive innovations: the change resulting from disruptive innovation has always been positive, leading to permanent improvement in the quality and quantity of the services and goods provided, and new entrants are the most likely to embrace and push disruptive innovations.

Healthcare has similarly seen and is continuing to see disruptive innovations. The following is a list of some of the disruptive innovations in healthcare in recent times and their future:

1. The development of consumer devices, wearable health trackers (e.g., the Apple Watch), and various applications that can be downloaded on a smartphone have disrupted and changed the health care industry significantly with far reaching repercussions. In the past, several parameters like pulse, heart rate, blood oxygen, and blood pressure could only be measured by professionals in a clinical setting. However, the advent of portable devices like pulse oximeter, smart watches and various health applications are able to measure these parameters instantaneously without hassle. Furthermore, these smart watch and smart phone applications also store these various measurement data which can be shared with a physician during a diagnosis.

2.  Artificial Intelligence (AI) and machine learning will also disrupt the medical field. Chatbots capable of answering commonly asked patient questions have become ubiquitous. AI can be used to collate and analyse survey responses. It has the potential to decrease healthcare costs and increase hospital efficiency exponentially.

3. Blockchain also has the potential to disrupt the healthcare market. It is defined as a growing list of records, called blocks, that are linked together using cryptography. It enhances privacy for online patient databases, and facilitates in storage of patient records, supply and distribution, research, etc.

4. Internet of Things (IoT) is a novel concept that can revolutionise healthcare. It is a development on the consumer wearables and applications mentioned in point 1. The IoT enables to store that huge mass of data in an online database, readily accessible by anyone with appropriate permissions at any time. It will allow ease of access to patient information, as well as help spread awareness and prevent pandemics by helping to identify Patient Zeroes of new virus strains.

5. Electronic Health Records have already revolutionised healthcare, especially prevalent in the USA after the passing of the Affordable Care Act. Not only is it helpful for patients, it also presents a huge chance for data analysis through AI and machine learning, presenting new business opportunities and venues of scientific research.

6. In 2019, Walmart formed Walmart Health, which are freestanding clinics that provide primary and urgent care. Similarly, Amazon also acquired the online pharmacy PillPack and CVS acquired Aetna.  All of these moves have created new giants in the industry, changing the way healthcare operates. 

\subsection*{Conclusion}

In a literature review published in late 2020 noted that the term and concept of "disruptive innovation" has become increasingly more common in scientific journals across the world, with North American journals predominating with 78.4\% of all such instances. The five most cited disruptive innovations are all "omics" technologies, mobile health applications, telemedicine, health informatics and retail clinics. As of now, in medical journals, the concept of disruptive innovations as such is very vague and inconsistent. However, the usage of the term has increased significantly, demonstrating the increased awareness around this term. One can only hope that more and more disruptive technologies are able to improve the healthcare industry significantly and make it more accessible, affordable and efficient.


\subsection*{References}

1. https://dhge.org/about-us/blog/disruptive-innovation-healthcare-examples

2. https://innovations.bmj.com/content/7/1/208


\end{document}