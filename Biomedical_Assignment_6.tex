\documentclass[11pt]{article}

\begin{document}

NAME: Soumil Thakurata

Roll No: 21111064

Stream: BIOMEDICAL ENGINEERING

Subject: Basic Biomedical Engineering

Topic: Assignment 6

\pagebreak


\section*{\centering Five Solutions to COVID-19 Provided by Biomedical Engineers}

Biomedical Engineers were instrumental in containing the COVID-19 pandemic. Various innovations were rapidly developed by such engineers, to contain the spread and reduce mortality. Following are some of the innovations pioneered by Biomedical Engineers for dealing with COVID-19

1. Biomedical engineers pioneered the development of Patient Isolation Hoods (PIHs) for the express purpose of protecting medical professionals in clinical settings. The PIHs cover a patient’s head and body, with apertures that allow doctors to insert a breathing tube or attach a ventilator as needed, and ample space within the structure to enable physical intervention by medical staff. Each design has two holes for doctors to insert their arms and access the patient, as well as an opening for a breathing tube. Other features include negative air pressure within to suck infectious aerosol particles away from the care-giving doctor, a lightweight disposable overall envelope, and no rigid form is placed on the patient to enhance comfort.

2. As development of rapid testing was essential for detection and containment, biomedical engineers developed COVID tests that were able to detect whether a patient was infected or not within hours, in stark contrast to the initial days or even weeks needed for test results. The new test, developed by biomedical engineers in UW–Madison’s AIDS Vaccine Research Laboratory, utilised reverse-transcriptase loop-mediated isothermal amplification (RT-LAMP) to amplify the identifiable parts of virus available in saliva samples for detection of COVID-19 infection. Unlike the time-consuming Polymerase Chain Reaction (PCR) test, RT-LAMP doesn't required specialised instrumentation. It is thus much more applicable to situations like schools, workplaces, etc. Furthermore, as RT-LAMP uses chemicals different from those used in the PCR test, it has helped ease the supply chain of those items due to the sudden surge in demand for PCR tests.

3. Biomedical engineers from UW-Madison developed a system of detecting COVID presence in populations by testing waste-water.They utilise a test called quantitative PCR, which relies on chemicals called primers and probes to search for and bind to the RNA specific to the COVID-19 virus. This was implemented with great success in Wisconsin in America, providing reliable data about the spread of the virus and reducing the need for irritating multiple tests on the population.

4. Biomedical Engineers also developed automated, contactless thermometers in order to reduce contact and the chance of infection. The contactless thermometer was created by reusing sensors from commercially available thermometers, and adding a microcontroller and distance sensor controlled by a computer code. In addition, the microcontroller has the ability to connect to the Internet. It was further developed and is currently based on a small, single-board computer called a Raspberry Pi and have additional functionalities.

5. Biomedical Engineers were critical to the production of Personal Protective Equipment (PPE), especially during the initial period of great shortage during the pandemic. Rebecca Alcock and her team developed a procedure for locally producing PPE in light of the global supply chain disruption during the pandemic. It included assessing needs, designing equipment, sourcing materials, retooling production lines, and matching manufacturers and healthcare providers, and was used with great effect initially in Guatemala. The small Central American country was able to distribute 75,000 PPE units within four weeks, an expansion in capacity so large it even caught the UNDP's attention.

\subsection*{Conclusion}

From the above paragraphs, one can see that biomedical engineers played an integral and invaluable part in containing the pandemic. They assisted in developing new techniques of detection, prevention of spread, as well as protection for medical personnel.  


\subsection{References}

1. https://www.gsd.harvard.edu/2020/04/gsd-begins-patient-isolation-hood-pih-design-and-fabrication-alongside-ongoing-ppe-efforts/

2. https://www.engr.wisc.edu/news/simpler-covid-19-test-could-provide-results-in-hours-from-saliva/

3. https://www.engr.wisc.edu/news/everybody-poops-some-shed-the-virus-that-causes-covid-19-wisconsins-wastewater-surveillance-is-looking-for-it/

4. https://www.engr.wisc.edu/news/at-uw-madison-contactless-thermometer-helps-monitor-student-health/

5. https://www.engr.wisc.edu/news/grad-student-helping-organize-ppe-production-covid-response-in-developing-countries/


\end{document}