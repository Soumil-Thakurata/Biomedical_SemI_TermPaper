\documentclass[11pt]{article}
\usepackage{graphicx}
\graphicspath{ {./images/} }

\begin{document}

{\centering PROJECT REPORT


On


Biomaterials and Its Applications


SUBMITTED BY:


Soumil Thakurata


MIS Roll Number: 21111064


B.Tech 1st Semester,


Biomedical Engineering 

National Institute of Technology, Raipur


\includegraphics{download (1)}

Under the Supervision of:

Saurabh Gupta Sir,

Biomedical Engineering Department

Biomedical Engineering \par}

\pagebreak

\section*{\centering Acknowledgement}

I am grateful to Saurabh Gupta Sir for his proficient supervision of the term project on “Biomaterials and Its Applications". I am very thankful to you sir for your guidance and support.
\\
\begin{flushright}
Soumil Thakurata

MIS Roll No: 21111064

1st Semester, Biomedical Engineering

National Institute of Technology, Raipur
\end{flushright}

Date of Submission: 25th March, 2022

\pagebreak
\section*{\centering Abstract}
In this term paper, we will discuss biomaterials, with special reference to its applications. We will discuss its widespread usage in various roles, its increasing usage as more innovations are made in this regard, construction of biomaterial based instruments and so on.
\pagebreak

\section*{Introduction}

The first historical use of biomaterials dates back to ancient Egypt, where animal sinews were used to make primitive sutures. Since then, this field has come a long way forward. The modern field of biomaterials combines physics, chemistry, biology, medicine, advances in tissue engineering and materials science to make new, innovative instruments. There are naturally quite a few restrictions on what can be called a biomaterial, because substances that can have deleterious effects on humans cannot be classified nor be used as such. Biomaterials are therefore classified into two discrete categories: natural or synthetic (man-made). Gelatin, heparin, hyaluronic acid, alginate, etc. fall under the natural category, while synthetic biomaterials mostly consist of biodegradable polymers. 

\subsection*{Classification of Biomaterials}

Biomaterials can be broadly classified into five categories:

1. Metals: These consist mainly of titanium and its alloys, gold, silver, stainless steal, etc. These metals have sufficient vigour, stiffness, tiredness resistance and tensile strength which gives them an advantage in being used as weight tolerating implants such as bone plates and pins, screws, dental root implants, wires, joint replacements, etc. However, their drawbacks are that they are susceptible to rust, difficult and complex to prepare, highly elastic, and have high density.

2. Ceramics: These consist of materials like alumina, zirconia and hydroxyapatite. Their main advantage lies in their eco-friendliness to living tissues, their sufficiently high vigour and firmness, and resistance to rust. They are also significantly lighter than their metal counterparts. They are thus used as covering for load bearing implants, medical sensors, and as dental and orthopaedic implants. However, some of their disadvantages are their hardness, inflexibility, relatively less tiredness resistance compared to their metal counterparts, and comparatively weaker mechanical vigour.

3. Polymers: These consist of materials like nylon, silicon and polyester. Similar to ceramics, their main advantages are their eco-friendliness to living tissue, sufficiently high resilience, high resistance to rust and simplicity of construction. They are mostly used for making contact lenses socket, heart valves, artificial blood vessels and hearts, hip joints etc. However, they do have less vigour compared to metals or ceramics, have a non-negligible chance of degradation, and are not firm in their shapes.

4. Bio-composites: Bio-composite biomaterials are physically powerful, have high resistance to rust, wear and tear, and are eco-friendly to living tissues. They are used in the roles of joint replacements, bone cement, dental implants, etc. However, their disadvantages are that require steadiness and homogeneity, and are highly difficult to construct.

\subsection*{Characteristics of Biomaterials}

Biomaterials must fulfil certain criteria before it can be used as such. It must be ensured that that particular material can last for a reasonably long time without the immune system have a reaction against it, or outright rejecting it (immune rejection). The characteristics are as follows:

1. It should have very good bio-compatibility.

2. Its mechanical properties should be well suited for its functions.

3. It should be sufficiently high quality both physically and chemically.

4. It should have sufficiently high resistance to wear and tear.

5. It should have sufficiently high resistance to rust.

6. For bone implants, osseo-integration is a necessity.

The above points are summarised in Figure 1.
\\
\\


\includegraphics{image1}
\\

Figure 1: Diagrammatic Representation of Design Requirements for Biomaterials [https://iopscience.iop.org/article/10.1088/1757-899X/1116/1/012178/pdf]
\\
\\
\subsection*{Applications of Biomaterials}

Biomaterial has wide applications in pharmaceuticals, food industry, the medical industry, and other household appliances. In the medical field, it has been used extensively in the medical industry, in applications like dental fixture fabrication, implants, prosthesis, tissue scaffolds, production of tablets and capsules, and for replacement of damaged or destroyed organs. A concise list of some applications of biomaterials is summarised as follows:

1. Substitution of a damaged part, for example, a kidney dialysis machine, hip joint and dental fixtures;

2. Support in the healing process, for example, bone plates, screws and sutures;

3. Support in prophylaxis, for example, catheters and drains;

4. Support in analysis, for example, catheters and probes;

5. Helps solve cosmetic problems, for example, in case of augmentation mammoplasty and chin augmentation;

6. Cures or solves functional defects, for example, using Harrington spinal rod to solve spinal defects;

7. Improving natural function of a body part, for example, cardiac pacemaker to ensure the heart keeps beating properly, and contact lenses to improve vision.

A more detailed application of biomaterials and its description is given below:

1. Cardiovascular biomaterials: Materials used to make implants like heart valves, cardiac pacemakers and vascular grafts which are used in the heart to reduce the intensity of cardiac disorders and diseases are called cardiovascular biomaterials. These are highly successful and are in high demand. The characteristics of the biomaterial used used depends on the type if impairment, for example, polymers are used in cardiac damage, although it has the disadvantage of limited blood compatibility in some cases. New technologies like surface modifications are therefore being researched and implemented to overcome this drawback.

2. Blood Vessel Vascular Grafts: Blood vessel prosthesis is the most common employed for the treatment of impaired blood vessels. Vascular bypass, more commonly knows as vascular graft technology is a surgical procedure which is utilised to minimise the impact of blockage of blood vessels. It is achieved by rejoining the blood vessels in such a manner that the blockage is bypassed. For vascular bypass, the most commonly used material is the patient's own vein (in case of autograft), or a donor's vein (in case of allograft). Some synthetic materials like dacron (polyethylene terephthalate), teflon (polytetrafluoroethylene) are also commonly used.

3. Heart valves: Valvular heart disease patients are usually treated with an artificial heart valve. This condition occurs whenever one or more of the four heart valves has an impairment in its function. The artificial heart valves have been classified into two categories, biological and mechanical, on the basis of their function. Biological valves are mostly made out of PTFE (teflon) and dacron. Mechanical heart valves on the other hand are made out of polymers, ceramics, titanium, silicone, as well as pyrolytic carbon. Transcatheter aortic implantation has also been achieved recently, by fabrication of metals, ceramics, polymers and pig heart valves.

4. Stents: The instrument used to bypass the blocked blood vessel is called a stent. They are inserted into the lumen of the blood vessel and can be either metallic or plastic. They can be further classified by function as urethral stents, coronary stents, prostatic stents, vascular stents, esophageal stents, and biliary stents. On the basis of physical and functional speciality, they can be further classified as bare-metal stents (BMS), drug-eluting stent (DES), and bio-absorbable stents.

5. Cochlear replacements: Patients suffering from sensorineural loss of hearing are usually treated with cochlear replacement. It is a  neuroprosthetic appliance that converts sound energy into electrical energy and then excites the auditory nerve. It is implanted via surgery. It consists of two parts: the sound processor which constitutes the exterior part, and contains microphones, electronics like DSP chips, battery and a coil, and the inner component that collects the signals and excites the cochlear nerve via a series of electrodes. Issues like biocompatibility, sterility and structural damage to tissues is key when implanting cochlear replacements.

6. Dental implants: Patients suffering from  bridge, crown, denture, and facial prosthesis usually undergo dental implantation process as a treatment. This dental implant process is also known as endosseous implant is highly successful and interfaces with the bone of the jaw or skull to support a dental prosthesis. Currently, osseo-integration is widely used for dental implants in which there is a process of bond formation that takes place between metal and the bone.T he implant fixture is permitted to osseo-integrate and then the dental prosthetic is added on top of it. Dental implants are most frequently made out of ceramics (, titanium and zirconium oxide, bioactive and
biodegradable ceramics), metals and alloys (titanium and titanium-6 aluminium-4 vanadium, cobalt-chromium-molybdenum-based alloy, iron-chromiumnickel-based alloys), or polymers and composites (PMMA, PE, PTFE, silicone rubber, polysulfone).  

7. Bone Cement: It is a biomaterial that is used for filling the gap between the bone and the prosthesis. It consists of two components: a powder (pre-polymerized PMMA/PMMA copolymer beads) that is the initiator and a liquid (MMA monomer), which is the inhibitor. The initiator is incorporated to the accelerator by the process of free radical polymerisation, such that the cement becomes a hard material. Bioceramics made up of CaP are used for bone regeneration due to their excellent biocompatibility, osseo-integration and osteoconduction. Most common example of CaP usage is in hip implants.

8. Joint replacement: Joint replacement surgery, or arthoplasty is a surgery  in which the impaired joint is replaced with an orthopaedic prosthesis. It is the only available and effective treatment for certain conditions like rheumatoid arthritis and osteoarthritis. Metal alloys like titanium alloys, Co-Cr alloys and stainless steel are commonly used as biomaterials for joint surgery. Preference is usually given to cobalt-chromiummolybdenum (CoCr or CoCrMo) and Titanium-6 aluminium-4 vanadium (Ti6Al4V) followed by titanium-aluminium-niobium (Ti6Al7Nb), iron-chromiumnickel (stainless steel, AISI 316 L) and commercially pure titanium (cpTi). Aluminium and zirconia are less commonly used biomaterials in this application.

9. Skin Tissue Substitutes: If, due to a medical condition or deep injury, natural skin regeneration does not take place, skin grafting is needed for treatment. Most commonly used synthetic skin substitutes are Biobrane, Alloderm, Integra and TransCyte. Biological skin substitutes are also available on the market. A few examples of biological skin substitutes with allogeneic cells are Dermagraft, OrCel, Apligraf, while an example of a biological skin substitute with autologous cells is Epicel.

10. Intraocular lenses (iols) for eye surgery: In this surgery opacified contents in the capsular bag of the cataractous lens are remove d and intraocular lenses are incorporated inside the capsular bag for re-establishing the normal refractive power. Biocompatibility is the key issue while carrying out this surgery. The lenses are usually made out of either acrylic or silicone. Acrylic lenses can be foldable or non-foldable, while silicone lenses are always foldable. 

\subsection*{\centering Conclusion}

We have thus seen that biomaterials has a wide range of vital usages in the medical industry presently. It has become very popular and is continuing to see increased usage across the world. We can only hope that continuing innovations will see further improvement and development of biomaterials.




\subsection*{References}

1.https://www.nibib.nih.gov/science-education/science-topics/biomaterials

2. https://iopscience.iop.org/article/10.1088/1757-899X/1116/1/012178/pdf



\end{document}