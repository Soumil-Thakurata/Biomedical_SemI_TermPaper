\documentclass[11pt]{article}

\begin{document}

NAME: Soumil Thakurata

Roll No: 21111064

Stream: BIOMEDICAL ENGINEERING

Subject: Basic Biomedical Engineering

Topic: Assignment 5

\pagebreak


\section*{\centering Emerging Technologies in Healthcare}

Medical technology has seen increasingly rapid technological progress in recent years. The development of modern computing systems, nuclear physics, artificial intelligence and machine learning have revolutionised medical treatments and the healthcare industry. The following is a list of some of the new emerging technologies in healthcare that will improve the industry significantly.

1. Remote Patient Monitoring (RPM): RPM has given physicians the opportunity to monitor their patients from afar and without being physically close to them. This has proven particularly helpful during the COVID-19 pandemic, as patients with relatively minor ailments were able to circumvent the risk of getting infected by the virus by visiting the hospital for consultation with a physician. RPM in fact goes hand in hand with telemedicine, reducing the need for patient travel and reducing exposure to diseases in the hospital. Spyglass Consulting Group conducted a survey which found out that 88$\%$ of healthcare providers had invested in or were evaluating adding RPM to their practice. Provenance, a provider of medical software, in 2020 estimated that about 23.4 million patients used some form of RPM.

2. Artificial Intelligence (AI): AI has been and is being increasingly being used to collect and analyse a vast amount of patient data in order to deliver tailored diagnoses and treatment recommendations to patients. This not only has reduced costs, but also increased the speed and efficiency with which physicians are able to diagnose a patient and recommend suitable medications (with appropriate dosages determined by the AI from the particular patient's past history) in order to achieve optimum healthcare delivery.

3. Digital Therapeutics: Digital therapeutics refer to sophisticated applications on a patient's smartphone or personal computer that can monitor, collect ans analyse relevant data of the patient subject and communicates it with the physician. This technology is especially useful for patients with chronic illnesses (for example, diabetes type I and type II, asthma, cancer, migraines, etc.) as it enables information on the patient's condition to be collected quickly and efficiently which is then analysed or interpreted by the concerned physician. This removes the need for constant visitation of the chronically ill patient to the hospital and repeated invasive testing of various parameters. These software programs go through the same rigorous testing as all medications, including randomized clinical trials.

4. Nanomedicine: Nanomedicine is the medical application of nanotechnology, the technology that operates on the atomic, molecular, or supramolecular scale. BlueWillow Biologics, a biopharmaceutical company, in 2021 developed nanotech that is able to fight against bacteria and viruses. CytImmune Sciences, a leader in cancer nanomedicine,  similarly has recently completed a Phase I trial of using gold nanoparticles to target drug delivery to tumours. 

5. Healthcare Digital Assistants: Digital applications like Alexa and Google Home have significantly improved customer interaction with technology. In 2020, the two largest designers of electronic health records systems, Epic and Cerner, began integrating voice-enabled virtual assistants on their systems. AI startup Saykara has also launched a new voice-assistant that can listen and comprehend a patient-physician conversation, and does not require voice commands to activate. 

6. Cancer Immunotheraphy: The premise of immunotherapy is to genetically alter a patient's cells to work in tandem with the body's immune system to fight cancer. The benefit of immunotheraphy is that side-effects are minimised as healthy cells aren't destroyed as collateral damage. New treatments using joint therapy and engineered T-cell have been developed which can create what will ideally be very effective therapies for a wide range of tumour profiles.  

A paper published in 2020 identified several new emerging technologies with potential care and support applications for older people. These were: ``1) assistive autonomous robots; (2) self-driving vehicles; (3) artificial intelligence-enabled health smart apps and wearables; (4) new drug release mechanisms; (5) portable diagnostics; (6) voice-activated devices; (7) virtual, augmented, and mixed reality; and (8) intelligent homes.'' This thus reinforces the increasing use of AI and machine learning in revolutionising healthcare and being the chief emerging technology. 

One can only hope that more work will continue to be done in order to further develop and commercialise these new technologies which will make healthcare more affordable, accessible and efficient for all. 


\subsection*{References}

1. https://www.medicaltechnologyschools.com/medical-lab-technician/top-new-health-technologies

2. https://pubmed.ncbi.nlm.nih.gov/32780020/

\end{document}