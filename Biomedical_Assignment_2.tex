\documentclass[11pt]{article}

\begin{document}

NAME: Soumil Thakurata

Roll No: 21111064

Stream: BIOMEDICAL ENGINEERING

Subject: Basic Biomedical Engineering

Topic: Assignment 2

\pagebreak

\section*{Evolution of Modern Health Care System}

The Health System in the world has changed rapidly from a system of simple home remedies and travelling doctors/quacks with little to no training into a complex industry involving massive advances in human's knowledge of anatomy, physics and chemistry. New instruments like MRI (Magnetic Resonance Imaging), PET (Positron Emission Tomography) scanner, X-Ray Machines have come into common usage and have allowed doctors and physicians to diagnose ever more accurately diseases inflicting their patients. Since the rise of the internet, newer forms of delivering treatments like telemedicine, personalised treatments have become more common.

Similarly, the delivery of healthcare has changed. For most of the twentieth century, healthcare had three facets: patients relied on autonomous physicians to act as their agents; patients received complex care from independent, non profit hospitals; and
insurers did not intervene in medical decision making and reimbursed physicians, hospitals, and other providers on a fee-for-service basis. This led to spiralling costs for patients and increased wastage of medical resources.  

This has been recently replaced by managed care, mainly large hospitals with huge resources which have been able to do the following: optimise healthcare through preventive medicine; reduce overutilization and unnecessary utilization of expensive services; standardise and control the widely varying quality of care offered by traditional fee-for-service providers. 

\subsection*{Personalised Medicine}

One of the more modern aspects of modern healthcare is personalised medicine. It is defined as 'a medical model that separates people into different groups—with medical decisions, practices, interventions and/or products being tailored to the individual patient based on their predicted response or risk of disease.' Personalised medicine, if further developed, has the promise of enabling each patient to receive earlier diagnoses, risk assessments and optimal treatments, while also reducing costs. 

As each person has an unique variation of the human genome, personalised medicine uses techniques including but not limited to genome wide association study (GWAS), and RNA-sequencing (RNA-seq) to help design preventive measures (for example, changing the lifestyle of a patient prone to diabetes genetically) to help prevent the incidence of conditions which their genetic make-up make the patients susceptible to. One example is the toxic effects of various drugs used to treat breast cancer. Oncologists today can choose a medication, such as trastuzumab (Herceptin, Genentech) based on standard drug therapy and dosing guidelines for that disease. They can also consider such factors as a patient’s weight, age, and medical history and the reactions of other blood relatives to the same drug. This helps prevent unnecessary side effects.

\subsection*{Telemedicine}

Telemedicine, or telehealth, is defines as 'distribution of health-related services and information via electronic information and telecommunication technologies.' Increased use of telehealth can reduce the chances of infections contracted due to visiting hospitals, reduces labour costs and increases the efficiency with which doctors treat their patients. Furthermore, telehealth does not affect the quality of the care or treatment given to patients. It also has the added benefit of reducing the costs of healthcare and making it more affordable for people.

Telehealth has played a vital role during the COVID-19 pandemic. Doctors carried out diagnosis of mild non-COVID-19 ailments online. It has been proven particularly useful for detecting glaucoma. A study conducted in the United Kingdom on 24,000 subjects showed an agreement of 87$\%$ between optometrist and ophthalmologist examinations. Similar results were obtained in Kenya, where teleglaucoma disclosed a sensitivity of 41.3$\%$ and a specificity of 89.6$\%$ as compared to the standard fundus oculus exam by the eye-care specialists.

\subsection*{Conclusion}

In short, health care has dynamically changed over the last few decades. Massive improvements have been observed, and further work is being conducted to make healthcare accessible, affordable and effective for all.

\subsection*{References}

1. https://www.ncbi.nlm.nih.gov/pmc/articles/PMC2957753/

2. Wikipedia, 'Personalised Medicine'

3. https://www.encyclopedia.com/science/encyclopedias-almanacs-transcripts-and-maps/evolution-us-healthcare-system

4. https://quod.lib.umich.edu/m/mfr/4919087.0007.102/--health-care-in-the-united-states-an-evolving-system

5. Wikipedia, 'Telehealth'

6. https://www.hindawi.com/journals/jdr/2020/9036847/





\end{document}