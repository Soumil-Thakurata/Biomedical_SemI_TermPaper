\documentclass[11pt]{article}

\begin{document}

NAME: Soumil Thakurata

Roll No: 21111064

Stream: BIOMEDICAL ENGINEERING

Subject: Basic Biomedical Engineering

Topic: Assignment 3

\pagebreak

\section*{\centering Future of Healthcare}

As the world emerges from the COVID-19 pandemic, the manner in which healthcare operates has dramatically changed to cope with the changing circumstances. The following trends are clearly evident: increased use of Artificial Intelligence in testing, digitisation of doctor consultations to reduce hospital visits, digitised patient information which will be stored in a cloud database, and increased use of portable variants of bulkier machines. The multinational corporation Deloitte has identified the following ten features of the future of healthcare:

1. Increased collection and usage of data as a basis of preventing and treating diseases in patients. Greater data will be collected on an institutional, individual, environmental, and population data and aggregating it. This will allow the development of better contingency plans to control epidemics, pandemics, etc.

2. Development of AI and software engines which develop powerful and useful models, generate analytical tools, and data insights to aid in healthcare research and improvement of patient health.

3. Development of online infrastructure to cater to the needs of the evolving healthcare industry. This will consist of development of new and better cloud databases, better encryption to protect patient privacy, and better offline infrastructure (such as better internet speeds) to transfer the process of healthcare online.

4. Healthcare product developers would have to diversify, as they would need to provide new products to the consumer health ecosystem by developing and manufacturing wellness and care products: from applications to drugs and devices. The economic models of such developers would have to change to ensure that they are able to sustainably and profitably maintain their ability to enable well-being and care delivery. The products won't be limited to pharmaceuticals and medical devices, but rather encompass software, applications and wellness products.

5. The market will see the rise of consumer-centric players. They will provide virtual, personalized wellness and care to consumers, encourage behavioural change, and drive education for both the consumers and the caregiver.

6. Improvement in specialist healthcare will also be observed. Specialty care operators will provide essential specialty care and interventions when in-home wellness and care efforts will be insufficient to deal with pernicious diseases or conditions.

7. Development of localised health hubs where common treatments will be carried out. It will serve as a centre for education, prevention and treatment in a retail setting. It will also connect with various virtual, home and auxiliary home wellness providers.

8. The logistical services required to maintain the improved healthcare system smoothly will also see improvements. They will operate the "just in-time" supply chains, device and medicine procurement operations, and deliver the product to the consumer.

9. The insurance packages offered by insurances will also change from generalised packages into personalised and tailored packages for their consumers to ensure efficiency, cost savings and general wellness of the consumer.

10. To cope with changing times, regulators and regulations will change. Their position will also change, that is, instead of playing the role of a policeman, they will actively encourage and steer the industry towards further improvements.

11. The future will see the development of biomarkers in order to more accurate predict the pernicious condition of a cardiac arrest, and potentially save many lives. 

12. Another important area of research will be into the field of regenerative disease, especially with the advent of new research in stem cells. This will ensure better recovery of persons with fractures, irreversible muscle damage, etc. 

\subsection*{Conclusion}

A paper in 2001 noted the following trends in healthcare over the next decade: more technology, more information, the patient as the ultimate consumer, development of a different delivery model, innovation driven by competition, increasing costs, increasing numbers of uninsured, less pay for providers, and the continued need for a new healthcare system. These predictions have been proven eerily accurate. We can only continue to hope and work on improving the current healthcare system to make it affordable, sustainable and efficient for all.

\subsection*{References}

1. https://www.ncbi.nlm.nih.gov/pmc/articles/PMC3116776/

2. https://www2.deloitte.com/us/en/pages/life-sciences-and-health-care/articles/future-of-health.html

\end{document}